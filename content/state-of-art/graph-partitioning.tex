\section{Graph Partitioning and Clustering Algorithms}
\label{sec:clusteralgorithms}

The task of finding clusters which are optimal with respect to fitness measures
is NP-complete \citep{sima2006np}. The following are two current algorithms,
time-linear to the number of edges, which try to find solutions through
approximation.

\subsection{Chinese Whispers.}
\label{sec:cw}

\ac{CW} is a randomised graph-clustering algorithm which is time-linear in the
number of edges, developed by Biemann \citep{biemann2006chinese}. It is a basic,
yet effective, algorithm to partition nodes of weighted, undirected graphs, and
it is said to perform well in small-world graphs.

\ac{CW} is parameter-free, as the number of partitions emerges naturally during
the process. But formally, \ac{CW} does not converge, as a node can become tied
and be randomly assigned a different class at each iteration, without ever
stabilising, nor is it deterministic, due to the random orders and assignments
in the algorithm.

In \ac{CW}, a weighted graph, $G=(V,E)$, has nodes $v_i \in V$ and weighted
edges $(v_i, v_j, w_{ij}) \in E$ with weight $w_{ij}$. If $(v_i, v_j, w_{ij})
\in E$ implies $(v_j, v_i, w_{ij}) \in E$ then $G$ is undirected. If all weights
are 1, $G$ is unweighted. The degree of a node is the number of edges it takes
part in. The neighbourhood of a node $v$ is the set of all nodes $v'$ such that
$(v,v',w) \in E$ or $(v',v,w) \in E$.

\ac{CW} works as outlined in Algorithm~\ref{alg:cw}. First, all nodes get
different classes. For a small number of iterations, in each iteration, the
nodes inherit the strongest class in their neighbourhood. This is the class
whose sum of edge weights to the current node is maximal. If multiple classes
are equally the strongest, one is chosen randomly. Classes are updated
immediately, a node can obtain classes introduced in that same iteration.

Regions of the same class stabilize during the iteration, and grow until they
reach the border of another class.

\begin{algorithm}
 \caption{The Chinese Whispers algorithm}
 \label{alg:cw}
 \begin{algorithmic}
  \Function{ChineseWhispers}{V, E}
   \ForAll{$v_i \in V$}
    \State $class(v_i) \gets i$
   \EndFor

   \While{changes}
    \ForAll{$v \in V$, random order}
     \State $class(v) \gets max\_rank(class(neighbourhood(v)))$
    \EndFor
   \EndWhile
  \EndFunction
 \end{algorithmic}
\end{algorithm}

Apart from ties, the class of a node usually does not change more than a few
times. The number of iterations depends on the larger distance between two nodes
in the graph.

\ac{CW} was evaluated on tasks of \ac{WSI} using a similar approach to the one
in \citep{dorow2003discovering}, by replacing the Markov Clustering algorithm
with \ac{CW}. The evaluation method in \citep{bordag2006word} was used. In this,
the neighbourhood of two words is merged and the ability of the algorithm to
separate the merged graph is evaluated. The evaluation measures used included
\textit{retrieval precision} ($rP$), the similarity of the found sense with the
gold standard sense using the overlap measure, and the \textit{retrieval recall
($rR$)}, which are the amount of words assigned correctly to the gold standard
sense.

Results for nouns showed that \ac{CW} had a retrieval precision of $94.8\%$ and
a retrieval recall of $71.3\%$, which suggest similar performance as specialized
graph-clustering algorithms for \ac{WSI} given the same input.

\subsection{MaxMax.}
\label{sec:maxmax}

MaxMax is a soft-clustering algorithm applicable to edge-weighted graphs
\citep{hope2013maxmax}. It is parameter-free, runs in linear time to the number
of edges and it is deterministic. Test results show it to return scores
comparable with existing state-of-the-art systems.

In MaxMax, a notion of \textit{maximal affinity} is used, in which affinity
between vertices $u$ and $v$ is the edge weight $w(u,v)$. A vertex $u$ has
maximal affinity to a vertex $v$ if $w(u,v)$ is maximal among all edges with
$u$. $v$ is said to be a \textit{maximal vertex of $u$}.

MaxMax consists of two stages, as described in Algorithm~\ref{alg:maxmax}.
First, the weighted graph $G$ is transformed in an unweighted, directed graph
$G'$. Maximal affinity relationships between vertices are used to determine edge
direction in $G'$. If in $G$ a vertex $u$ has two maximal vertexes $v$ and $w$,
in $G'$ $u$ will have only two directed edges, from $v$ to $u$ and from $w$ to
$u$.

Then, clusters are identified, finding the \textit{root} vertices of subgraphs
of $G'$ and by marking all descendants of a vertex as $\neg root$. In the
directed graph $G'$, a vertex $v$ is a \textit{descendant} of $u$ is there is a
directed path from $u$ to $v$. At the end of the stage, vertices which are still
marked as $root$ uniquely identify clusters.

\begin{algorithm}
 \caption{The MaxMax algorithm}
 \label{alg:maxmax}
 \begin{algorithmic}
  \Function{MaxMax}{G=(V, E)}
   \State $G' = (V, E') \gets$ a directed graph where $(v,u) \in E'$ iff
   $(u,v) \in E$ and $v$ is a maximal vertex of $u$

   \State Mark all vertices of $G'$ initially as \textit{root}

   \ForAll{$v \in V$}
    \If{$v$ is $root$}
     \ForAll{$u \in descentant(v) \setminus v$}
      \State $u \gets \neg root$
     \EndFor
    \EndIf
   \EndFor
  \EndFunction
 \end{algorithmic}
\end{algorithm}

MaxMax was evaluated in the context of the SemEval 2010 \ac{WSI} Task
\citep{manandhar2010semeval}, using an adaptation of the Shared Nearest
Neighbours algorithm and then using MaxMax to identify sense clusters in the
generated target word graph. MaxMax has the best scoring V-measure (of $32.8\%$)
among the systems evaluated in \citep{manandhar2010semeval}, and the worst
F-score (of $13.2\%$) among the systems evaluated. The authors claim that their
system is overly penalized by the F-score due to the way this is known to be
biased towards clustering solutions returning large clusters and to punish small
clusters disproportionately.

% kate: default-dictionary en_GB; indent-width 2; replace-tabs on;
% kate: remove-trailing-space on; space-indent on;
% kate: replace-trailing-space-save on; remove-trailing-space on;
