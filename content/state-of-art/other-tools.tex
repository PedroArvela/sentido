\section{Other Tools}

In addition to the existing work in \ac{WSD} and \ac{WSI}, it is relevant to
mention other works which make use of word co-occurrences in text.

\subsection{STRING}
\label{sec:string}

\ac{STRING} \cite{mamede2012string} is a \ac{NLP} system for the Portuguese
language, capable of preforming all basic \ac{NLP} tasks, including \ac{NER},
\ac{IR}, \ac{AR}, among other tasks. \ac{STRING} is composed of several stages,
as described in Figure~\ref{fig:stringarch}.

\begin{figure}[ht]
 \caption{\acs*{STRING} Architecture}
 \label{fig:stringarch}
 \centering
 \tikzstyle{block} = [rectangle, draw, node distance=1.5cm,
    text width=2.5cm, text centered, minimum width=3cm, minimum height=3em,
    rotate=90, anchor=base]
\tikzstyle{line} = [draw, -latex']

\begin{tikzpicture}[node distance = 2cm, on grid]

  \node [block] (lexman) {LexMan};
  \node [block, below of=lexman] (rudrico) {RuDriCo2};
  \node [block, below of=rudrico] (marv) {MARv4};
  \node [block, below of=marv] (xip) {XIP};
  \node [block, below of=xip] (anaphora) {Anaphora Resolution};
  \node [block, below of=anaphora] (time) {Time Normalization};
  \node [block, below of=time] (slot) {Sloft Filling};
  \node [block, below of=slot] (event) {Event Ordering};
  \node [block, below of=event, ellipse, text width=1cm] (join) {Join};

  \path [line] ([xshift=-0.5cm]lexman.north) -- (lexman);

  \path [line] (lexman) -- (rudrico);
  \path [line] (rudrico) -- (marv);
  \path [line] (marv) -- (xip);

  \path [line] (xip) -- ([yshift=1cm]xip.east) --
               ([yshift=1cm]anaphora.east) -- (anaphora);
  \path [line] (xip) -- ([yshift=1cm]xip.east) --
               ([yshift=1cm]time.east) -- (time);
  \path [line] (xip) -- ([yshift=1cm]xip.east) --
               ([yshift=1cm]slot.east) -- (slot);
  \path [line] (xip) -- ([yshift=1cm]xip.east) --
               ([yshift=1cm]event.east) -- (event);
  \path [line] (xip) -- ([yshift=1cm]xip.east) --
               ([yshift=1cm]join.east) -- (join);

  \path [line] (anaphora) -- ([yshift=-1cm]anaphora.west) --
               ([yshift=-1cm]join.west) -- (join);
  \path [line] (time) -- ([yshift=-1cm]time.west) --
               ([yshift=-1cm]join.west) -- (join);
  \path [line] (slot) -- ([yshift=-1cm]slot.west) --
               ([yshift=-1cm]join.west) -- (join);
  \path [line] (event) -- ([yshift=-1cm]event.west) --
               ([yshift=-1cm]join.west) -- (join);

\end{tikzpicture}

% kate: default-dictionary en_GB; indent-width 2; replace-tabs on;
% kate: remove-trailing-space on; space-indent on;
% kate: replace-trailing-space-save on; remove-trailing-space on;

\end{figure}

\subsubsection*{Preprocessing.}

In the first stage, the text is passed through a tokenizer, which is also
responsible for recognising some special tokens, such as numbers or punctuation.
The tokens are then passed through a \ac{POS} tagger, LexMan
\cite{vicente2013lexman}, which uses finite state transducers to label tokens
with their appropriate \ac{POS} tags. At last, the text is split into sentences
based mainly on the punctuation and having into account abbreviations, acronyms,
and the use of ellipsis.

\subsubsection*{POS Disambiguation.}

This stage comprises two steps, which are preformed by two distinct modules:

\begin{enumerate}
 \item \ac{RuDriCo}, which preforms rule-driven morphossyntactic disambiguation
\cite{diniz2010conversor};
 \item \ac{MARv}, which preforms statistical disambiguation
\cite{ribeiro2003anotacao}.
\end{enumerate}

\ac{RuDriCo} is also responsible for modifying the initial segmentation done by
the previous module, through the use of declarative pattern-matching rules. It
preforms the expansion of contracted words.

\ac{MARv} is responsible for looking at the labels generated for each word
and then choosing the most likely tag given its immediate context. This is done
using \acp{HMM}. \ac{MARv} uses second-order models, which codify contextual
information concerning entities, and unigrams, which codify lexical information.

\subsubsection*{Syntactic Analysis.}

Syntactic analysis is done through the use of \ac{XIP}, which also allows for
adding lexical, syntactic and semantic information to the output of the
previous modules. \ac{XIP} is responsible for several tasks:

\begin{enumerate}[label=(\roman*)]
 \item It adds information to the existing tokens through the use of lexicons.
\ac{XIP} includes a pre-existing lexicon that can be both extended or modified;
 \item It allows to define local grammars using pattern-matching rules, to group
lexical units together into a single entity;
 \item It also preforms a shallow parsing to group elements into chunks (noun
 phrase -- NP, prepositional phrase -- PP, among others) using linear precedence
 and sequence rules;
 \item Finally it preforms a deep parsing to extract the dependencies between the
different chunks (subject, direct object, modifier, etc.).
\end{enumerate}

\subsubsection*{Post-Syntactic Analysis.}

At this stage additional tasks are executed which extract additional information
from the text. One is the \ac{AR} task \cite{marques2013anaphora}, which
preforms anaphora identification in the text and then uses the \ac{EM} algorithm
to obtain the probabilities of each antecedent candidate. Other tasks performed
include time normalisation \cite{mauricio2011normalizacao} and event ordering
\cite{cabrita2014ordenar}.

\subsection{Syntax Deep Explorer}

In \cite{correia2015syntax}, a tool was developed to make use of \ac{STRING} to
allow one to explore co-occurrence data obtained from Portuguese texts. The
presented solution was composed of a tool to extract co-occurrences and
calculate the association measures of these as well as a web-based interface to
display these co-occurrences in an intuitive fashion.

The developed solution takes advantage of the rich lexical resources extracted
and the syntactic and semantic analysis of \ac{STRING} to provide information
about the patterns of co-occurrences found in the corpora evaluated.

\subsubsection*{Dependencies and Properties}

For the project in question, each co-occurrence is considered a syntactic
dependency extracted from \ac{XIP}. These dependencies have two words, in which
the first word is the modified element and the second word is the modifier,
information about the dependency itself, and a set of properties, if these
exist.

The project only considers the a subset of possible properties as relevant for
its goals, which indicate if the modifier is before, or after the modified word
or if the modifier is a focus adverb.


\subsubsection*{Architecture}

The implemented solution split the problem into four separate components:

\begin{itemize}
 \item The storage format of the extracted and computed information;
 \item The co-occurrence extraction from the corpus;
 \item The calculation of the various association measures;
 \item The display of the information in an user-friendly form.
\end{itemize}

\subsubsection*{Storage Format}

An SQLite database was chosen as the format to store the obtained information.
An \ac{ER} model was developed to represent the database, and a relational model
was generated from the created \ac{ER} model.

To store the information in the database, the \ac{ER} model in
Figure~\ref{fig:correira2015er} was created and used.

\begin{figure}[ht]
 \caption[ER model of (Correia et al. 2015)]{The \ac{ER} model used in
 \cite{correia2015syntax}.}
 \label{fig:correira2015er}
 \centering
 \tikzstyle{every node}=[font=\small]
\tikzstyle{every attribute}=[font=\small]

\begin{tikzpicture}[node distance = 2cm, on grid]

  \node [entity] (corpus) {Corpus}
    child {node [attribute, above of=corpus] (csource) {source}}
    child {node [key attribute, left of=csource] (cname) {\underline{name}}}
    child {node [attribute, below of=cname] (cgenre) {genre}}
    child {node [attribute, right of=csource] (cyear) {year}}
    child {node [attribute, below of=cgenre] (cupdate) {update}};

  \node [entity, line width=0.8pt, double, double distance=1pt, right of=corpus, node distance=9cm] (sentence) {Sentence}
    child {node [key attribute, above of=sentence] (sfile) {\underline{file}}}
    child {node [key attribute, left of=sfile] (snum) {\underline{sentenceNum}}}
    child {node [attribute, right of=sfile] (ssentence) {sentence}};

  \node [entity, below of=corpus, node distance=6cm] (word) {Word}
    child {node [key attribute, below of=word] (wid) {\underline{wordId}}}
    child {node [attribute, left of=word] (wclass) {class}}
    child {node [attribute, below of=wclass] (wword) {word}};

  \node [entity, below of=sentence, node distance=3cm] (dependency) {Dependency}
    child {node [key attribute, above of=dependency] {\underline{dependencyType}}};

  \node [entity, line width=0.8pt, double, double distance=1pt, below of=dependency, node distance=3cm] (property) {property}
    child {node [key attribute, below of=property, node distance=1cm] (ptype) {\dashuline{propertyType}}};

  \node [relationship, right of=corpus, node distance=5cm] {has}
    edge [-latex] (corpus)
    edge (sentence);

  \node [relationship, below of=corpus, node distance=2cm] (belongs) {belongs}
    child {node [attribute, below left of=belongs] (bfreq) {frequency}}
    edge (corpus)
    edge (word);

  \node [relationship, double, below of=dependency, node distance=1cm] {has}
    edge (dependency)
    edge (property);

  \node [relationship, right of=word, node distance=4cm] (cooc) {Co-Occurrence}
    edge (property)
    child {node [attribute, below left of=cooc, xshift=-0.7cm, yshift=0.7cm] (freq) {frequency}}
    child {node [attribute, below of=freq, node distance=0.8cm] (pmi) {pmi}}
    child {node [attribute, right of=pmi, node distance=0.8cm, yshift=-0.5cm] {dice}}
    child {node [attribute, below of=cooc] {logdice}}
    child {node [attribute, below of=cooc, xshift=2cm, yshift=-0.1cm] {chipearson}}
    child {node [attribute, below of=cooc, xshift=2.2cm, yshift=0.7cm] {loglikelihood}}
    child {node [attribute, right of=cooc, xshift=0.4cm, yshift=-0.5cm] {significance}};

  \path [draw] (property) -- (belongs);

  \node [draw,dashed, minimum width=15cm, minimum height=4cm] (agg) at ([yshift=-0.6cm]cooc) {};
  \node [below right of=ptype] {aggregation};

  \draw [-latex] (cooc) -- (corpus);
  \draw (cooc) -- (word);
  \draw (cooc) -- (2cm,-5.5cm) -- (word);

  \node [relationship, above of=cooc, node distance=4cm] {Exemplifies}
    edge (sentence)
    edge (agg);

\end{tikzpicture}

\end{figure}

The entities \textit{Corpus}, \textit{Word} and \textit{Dependency} are used to
store the information about each corpus, word and dependency type respectively.

The weak entity \textit{Property} associates the \textit{Dependency} with a
property type.

Each word has a \textit{Belongs} relation with the corpus, which indicates how
often the word occurs in the Corpus.

Each pair of words and property also have a \textit{Co-occurrence} relation,
with a \textit{frequency} attribute which defines how often these words occur
together in the same corpus with the given property. Additionally, several
attributes exist for each kind of association measure.

The weak entity \textit{Sentence} has a \textit{Belongs} association with the
\textit{Corpus}, and the aggregation of the \textit{Co-occurrence} associations
has a \textit{Exemplifies} association with the \textit{Sentence} entity.

Additionally, the following \acp{IC} were identified:

\begin{enumerate}
  \item The words in the \textit{Co-occurrence} association must belong to the
\textit{Corpus} to which they are associated with.
  \item The \textit{Co-occurrence} association must be associated to the same
\textit{Property} with which the words associate with in \textit{Belongs}.
  \item The sentences in the \textit{Exemplifies} association must belong to the
\textit{Corpus} to which the given \textit{Co-occurrence} is associated with.
\end{enumerate}

\subsubsection*{Co-occurrence extraction}

The co-occurrence extractor obtains the processed XML from XIP, and then parses
it to obtain the dependency information.

The extractor reads all dependencies parsed from the XML and stores in the
database, for each:

\begin{itemize}
  \item The type of the dependency in question;
  \item The lemma and class of each word in the dependency;
  \item The property of the dependency;
  \item The sentence in which the dependency occurred.
\end{itemize}

\subsubsection*{Association Measures}
\label{sec:assoc}

After the database is populated with the co-occurrences from the \textit{corpus}
the association measures are calculated in batches of 2,000 co-occurrences
each.

The measures calculated are:

\begin{itemize}
  \item \ac{PMI};
  \item \textit{Dice} Coefficient;
  \item \textit{LogDice};
  \item Pearson's Chi-Square Test;
  \item Log Likelihood Ratio;
  \item Significance Measure.
\end{itemize}

\subsection{Information Display}

The extracted information which was stored in the database is then displayed to
the user through the use of a web interface written in \ac{PHP} and AngularJS.
The \emph{front-end}, executed on the \emph{client-side}, makes \ac{AJAX}
requests to the \emph{back-end}, executed on the \emph{server-side}.

% kate: default-dictionary en_GB; indent-width 2; replace-tabs on;
% kate: remove-trailing-space on; space-indent on;
% kate: replace-trailing-space-save on; remove-trailing-space on;
