\section{Other Tools}

In addition to the existing work in \ac{WSD} and \ac{WSI}, it is relevant to
mention other works which make use of word co-occurrences in text.

\subsection{Syntax Deep Explorer}

In \cite{correia2015syntax}, a tool was developed to make use of \ac{STRING} to
allow one to explore co-occurrence data obtained from Portuguese texts. The
presented solution was composed of a tool to extract co-occurrences in and a
web-based interface to display these in an intuitive fashion.

The developed solution takes advantage of the rich lexical resources extracted
and the syntactic and semantic analysis of \ac{STRING} to provide information
about the patterns of co-occurrences found in the corpora evaluated.

\subsubsection{Architecture}

The implemented solution split the problem into four separate components:

\begin{itemize}
 \item The storage format of the extracted and computed information;
 \item The co-occurrence extraction from the corpus;
 \item The calculation of the various association measures;
 \item The display of the information in an user-friendly form.
\end{itemize}

\subsubsection{Storage Format}

A database was chosen as the format to store the obtained information. An
\ac{ER} model was developed to represent the database, and a relational model
was generated from the created \ac{ER} model.

\begin{figure}
 \centering
 \tikzstyle{every node}=[font=\small]
\tikzstyle{every attribute}=[font=\small]

\begin{tikzpicture}[node distance = 2cm, on grid]

  \node [entity] (corpus) {Corpus}
    child {node [attribute, above of=corpus] (csource) {source}}
    child {node [key attribute, left of=csource] (cname) {\underline{name}}}
    child {node [attribute, below of=cname] (cgenre) {genre}}
    child {node [attribute, right of=csource] (cyear) {year}}
    child {node [attribute, below of=cgenre] (cupdate) {update}};

  \node [entity, right of=corpus, node distance=8cm] (sentence) {Sentence}
    child {node [key attribute, above of=sentence] (sfile) {\underline{file}}}
    child {node [key attribute, left of=sfile] (snum) {\underline{sentenceNum}}}
    child {node [attribute, right of=sfile] (ssentence) {sentence}};

  \node [entity, below of=corpus, node distance=6cm] (word) {Word}
    child {node [key attribute, below of=word] (wid) {\underline{wordId}}}
    child {node [attribute, left of=word] (wclass) {class}}
    child {node [attribute, below of=wclass] (wword) {word}};

  \node [entity, below of=sentence, node distance=3cm] (dependency) {Dependency}
    child {node [key attribute, above of=dependency] {\underline{dependencyType}}};

  \node [entity, line width=0.8pt, double, double distance=1pt, below of=dependency, node distance=3cm] (property) {property}
    child {node [key attribute, below of=property, node distance=1cm] (ptype) {\dashuline{propertyType}}};

  \node [relationship, right of=corpus, node distance=4cm] {has}
    edge [->] (corpus)
    edge (sentence);

  \node [relationship, below of=corpus, node distance=2cm] (belongs) {belongs}
    edge (corpus)
    edge (word);

  \node [relationship, double, below of=dependency, node distance=1cm] {has}
    edge (dependency)
    edge (property);

  \path [draw] (property) -- (belongs);

  \draw [dashed] ($(wclass.north west) + (-0.5,0.5)$) rectangle ($(ptype.south east) + (1.5,-1.3)$);
  \node [below right of=ptype] {aggregation};

\end{tikzpicture}

 \caption[ER model of (Correia et al. 2015)]{The \ac{ER} model used in
 \cite{correia2015syntax}.}
 \label{fig:correira2015er}
\end{figure}


% kate: default-dictionary en_GB; indent-width 2; replace-tabs on;
% kate: remove-trailing-space on; space-indent on;
% kate: replace-trailing-space-save on; remove-trailing-space on;
