\section{Co-occurrence Graph Models}
\label{sec:co-occurrence}

In graph-based models, the meanings of a word are represented by a weighted
and undirected co-occurrence graph. The nodes (or vertices) of the graph are the
words which occur in the corpus and the transitions (or edges) are 
co-occurrences, with the weight of the edge describing the amount of times that 
co-occurrence exists. Two words are said to co-occur if they both occur within 
the same context.

These models are based on the idea that co-occurrence graphs have the properties
of \textit{small world networks} \cite{veronis2004hyperlex}, that is, most nodes
are not neighbours to one another, but most nodes can be reached by a small
number of steps, as shown in Figure~\ref{fig:small-world}. These properties
then allow to search for highly interconnected bundles of co-occurrences, that
is, \textit{high density components}, which correspond to the senses being
searched.

\begin{figure}[ht]
 \caption{An example of a small world network.}
 \label{fig:small-world}
 \centering
 \includegraphics[width=0.5\textwidth]{graphics/small_world}
\end{figure}

\subsection{Graph Model for Unsupervised Lexical Acquisition}

A graph from a \ac{POS}-tagged corpus is made in \citep{widdows2002graph}, using
words as nodes and the grammatical co-occurrence relationships between pairs as
edges. The relationships extracted are the co-occurrences of Noun-Verb,
Verb-Noun, Adjective-Noun, Noun-Noun. To generate the edges the top-$n$
neighbours of each word are selected and turned into edges.

An incremental algorithm is then used to extract categories based on a given
word using affinity scores, which give more importance to words that are linked
to the existing neighbours. The algorithm is tailored to avoid infections from
spurious co-occurrences, preventing spurious links from relating to genuine
semantic similarity. The process to select and add the most similar node to a
set of nodes is described in Algorithm~\ref{alg:similarnodes}, where $A$ is a
set of nodes and $N(a)$ are all the nodes which are linked to a node $a$.

\begin{algorithm}
 \begin{algorithmic}
  \State $N(A) \gets \cup_{a \in A} N(a)$
  \State $b \gets u \in N(A) \setminus A, max(\frac{|N(u) \cap N(A)|}{|N(u)|})$
  \State $A \gets A \cup b$
 \end{algorithmic}
 \caption{\label{alg:similarnodes} Select the most similar node}
\end{algorithm}

The model was built using the 
\ac{BNC}\footnote{\url{http://www.natcorp.ox.ac.uk/}, last accessed on 
\DTMdate{2017-09-28}.} and evaluated against classes in WordNet. A WordNet 
class was considered to be the collection of synsets subsumed by a parent 
synset, for example, the class \textit{musical instruments} was the collection 
of all synsets subsumed by the WordNet \textit{musical instruments} synset. For 
a given seed word, the algorithm in \ref{alg:similarnodes} was used 
to find the $n$ nodes most closely related to it. Ten classes were chosen 
beforehand, and for each class 20 words were retrieved using a single seed-word 
from the class in question.

The results show that of a total of 200 retrieved words, only 36 were
incorrect, giving an accuracy of $82\%$.

\subsection{Word Sense Induction Using Graphs of Collocations}
\label{sec:collocations}

A graph of collocations is used in \citep{klapaftis2008word} to generate a
taxonomy of senses which takes into consideration word polysemy. In this, each
node corresponds to an occurrence of two words in the same context window (which
in the model is a paragraph) and two nodes are connected if two collocations
occur in the same context window.

The base corpus, $bc$, consists of paragraphs containing the target word, $tw$.
Besides $bc$, there is also a large reference corpus, $rc$. The project focuses
in inducing the sense of $tw$ given $bc$ as the only input.

At first, paragraphs with $tw$ are removed from $bc$ and all paragraphs from
$bc$ and $rc$ are \ac{POS} tagged. From these, only nouns are kept and
lemmatised.

Log-likelihood ($G^2$) \citep{dunning1993accurate} is then used to filter 
common nouns which are not semantically related to $tw$, by checking if a given 
word $w_i$ has a similar distribution in $bc$ and $rc$. If that is true, $G^2$ 
will have a small value and $w_i$ should be removed from $bc$.

The noun frequencies of $bc$ are stored in a list $lbc$ and the noun
frequencies of $rc$ are stored in a list $lrc$. For each word $w_i \in lbc$ is 
created a table of observed counts OT taken from $lbc$ and $lrc$, and a table 
of expected values ET under the model of independence. $G^2$ is then calculated 
using the equations in Subsection~\ref{subsec:loglikeliood}.

The $lbc$ list is then filtered by removing words with a smaller relative 
frequency in $lbc$ in relation to $lrc$. The resulting $lbc$ list is then 
sorted by the $G^2$ values. Words that have a $G^2$ smaller than a 
pre-specified threshold $p_1$ are then removed from $bc$. By the end of this 
process, each paragraph in $bc$ is converted to a list of nouns assumed to be 
topically related to $tw$.

With the base corpus now processed, collocations of two nouns are detected by
generating all $\binom{n}{2}$ combinations for each $n$-length paragraph.
Conditional probabilities (Equation~\ref{eq:pij}) are then used to generate the
weights of each collocation. Collocations that have a frequency and weight
higher than a pre-specified threshold are then used to generate the nodes of the
graph $G$.

\begin{equation}\label{eq:pij}
 p(i|j) = \frac{f_{ij}}{f_{j}}
\end{equation}

The constructed graph $G$ is sparse. A smoothing technique is applied to
discover new edges between vertices and to assign weight to all of the graph's
edges. For each vertex $i$, a vertex vector $VC_i$ is assigned containing the
vertices which share an edge with $i$ in $G$. The similarity between each $VC_i$
and $VC_j$ is then calculated using the \ac{JC} (Equation~\ref{eq:jc}).

\begin{equation}\label{eq:jc}
 JC(VC_i, VC_j) = \frac{|VC_i \cap VC_j|}
                       {|VC_i \cup VC_j|}
\end{equation}

The final graph, $G'$, is then clustered using the \ac{CW} algorithm, further 
described in Section~\ref{sec:cw}. This algorithm was used because it does not 
require input parameters and performs in linear time to the number of edges, 
although it is not guaranteed to converge.

\ac{WSD} is at last made by assigning one of the induced clusters to each
instance of the target word. For each target word in a given paragraph, a score
for each induced cluster is applied, based on the number of collocations which
occur in the paragraph.

To preform \ac{WSD}, for each paragraph of $bc$, each induced cluster is
assigned a score equal to the number of collocations occurring in it.

To evaluate the model, the framework of the \ac{SWSI} \citep{agirre2007semeval}
was used. The test corpus consists of texts from the Wall Street Journal
corpus, hand-tagged with OntoNotes senses \citep{hovy2006ontonotes}.

The model was evaluated under \textit{unsupervised evaluation} and 
\textit{supervised evaluation} (see above). In the unsupervised evaluation, the 
model achieved $88.6\%$ purity, $31\%$ entropy (these measures are described 
below, in Section~\ref{sec:unsupeval}) and an F-Score of $78\%$.

\subsection{UoY: Graphs of Unambiguous Vertices}

The model developed by \citet{korkontzelos2010uoy} is a relaxed version of the
model in \citep{klapaftis2008word}, described in Section~\ref{sec:collocations},
in which a node is generated from a single word if this word is considered 
unambiguous, otherwise, a node is only generated from a set of two words.

The corpus is first preprocessed, with the aim of capturing words contextually
related to the target. Sentences or paragraphs, \textit{snippets}, which contain
the target word are lemmatised and \ac{POS} tagged using the GENIA tagger. Only
nouns are kept and words which occour in a stoplist are filtered out. Nouns
which are infrequent in the reference corpus are removed and the log-likelihood
ratio ($G^2$) is used to compare the distribution of each noun to its
distribution in the reference corpus. If a noun's $G^2$ is lower than a
specified threshold, or if the noun has a higher relative frequency in the
reference corpus than in the target corpus, then that noun is removed. At this
stage, each snippet is a list of lemmatised nouns contextually related to the
target word.

The graph is constructed by first representing all nouns in the list as graph
vertices. Each noun within a snippet is combined with every other, generating
$\binom{n}{2}$ pairs. $G^2$ is applied once again to the pairs.

To filter out pairs that refer to the same sense, a vector with the snippet IDs
in which they occur is generated for each pair and each noun. A pair is
discarded if its vector is similar to both vectors of their component nouns,
using for that purpose the Dice coefficient \citep{dice1945measures}, which is 
later described in Subsection~\ref{subsec:dice}.

Edges are drawn based on the co-occurrence of the corresponding vertices in
snippets. The weight of the edge is the maximum of the conditional probabilities
of its vertices, calculated according to Equation~(\ref{eq:wab}), and lowly 
weighted edges are filtered out.

\begin{equation} \label{eq:wab}
 w_{a,b} = \frac{1}{2} \left( \frac{f_{a,b}}{f_a} + \frac{f_{a,b}}{f_b} \right)
\end{equation}

The graph is then clustered using \ac{CW}, described in Section~\ref{sec:cw}.
To reduce the number of clusters, a post-processing stage is applied, in which,
for each cluster $l_i$, a set of all snippets $S_i$ containing at least one
vertex of $l_i$ is generated. For any clusters $l_a$ and $l_b$, if
$S_a \subseteq S_b$  or $S_a \supseteq S_b$, these are merged.

The model was submitted for SemEval-2010 Task 14 \citep{manandhar2009semeval}
and evaluated using an unsupervised and a supervised method. The first was 
measured according to the V-measure and F-score, while the second was measured 
using recall. Parameters were tuned by choosing maximum supervised recall, 
resulting in threshold frequencies of 10, $G^2$ threshold of 10, collocations 
weights of 0.4 and similarity threshold for pairs-of-nouns vertices of 0.8.

Results showed that the model achieved results of a V-measure of $20.6\%$, an
F-score of $38.2\%$ on the unsupervised evaluation with nouns, and a supervised 
recall of $59.4\%$ on the supervised evaluation.

% kate: default-dictionary en_GB; indent-width 2; replace-tabs on;
% kate: remove-trailing-space on; space-indent on;
% kate: replace-trailing-space-save on; remove-trailing-space on;
