\chapter{Evaluation}
\label{ch:eval}

\chapter{Evaluation Methodology}
\label{ch:eval-method}

The evaluation is composed of two methods, an unsupervised evaluation and a
supervised evaluation. The unsupervised evaluation is used to assess the
resulting clusters' similarity to the \ac{GS} senses. The supervised evaluation
is used as an application-oriented assessment of the resulting clusters in the
task of \ac{WSD}.

\section{Unsupervised Evaluation}
\label{sec:unsupeval}

In this evaluation method, the set of resulting clusters is compared to a
\ac{GS}. This comparison is made by evaluating the clusters'
\textit{homogeneity} and \textit{completeness}. Homogeneity refers to the degree
that each cluster consists of data points primarily belonging to a single
\ac{GS} class. Completeness refers to the degree that each \ac{GS} class
consists of data points assigned to a single cluster
\citep{manandhar2009semeval}. To evaluate homogeneity and completeness, the
F-score and the V-measure will be used.

Given a particular \ac{GS} sense $gs_i$ of size $a_i$ and a cluster $c_j$ of
size $a_j$, the F-score of $gs_i$ and $c_j$ is the harmonic mean of its
precision and its recall, as defined in Equation~\ref{eq:smallf}
\citep{agirre2007semeval}. Precision of a class $gs_i$ with respect to cluster
$c_j$ is the number of common instances divided by total cluster size, i.e.
$P(gs_i, c_i) = \frac{a_{ij}}{a_j}$. The recall of a class $gs_i$ with respect
to cluster $c_j$ is the number of common instances divided by total sense size,
i.e. $R(gs_i, c_j) = \frac{a_ {ij}}{a_i}$.

\begin{equation} \label{eq:smallf}
 f(gs_i, c_j) = \frac{2P(gs_i,c_j)R(gs_i,c_j)}
                     {P(gs_i,c_j) + R(gs_i,c_j)}
\end{equation}

The F-score of a class $gs_i$ is the maximum F-score obtained at any cluster, as
according to Equation~\ref{eq:fscoregs}. The F-Score of the entire clustering
solution is the weighted average of the F-scores of each \ac{GS}, as in
Equation~\ref{eq:fscore}, where $q$ is the number of \ac{GS} senses and $N$ is
the total number of target word instances.

\begin{equation} \label{eq:fscoregs}
 F(gs_i) = \underset{c_j}{\max} f(gs_i, c_j)
\end{equation}

\begin{equation} \label{eq:fscore}
 FS = \sum_{i=1}^q \frac{|gs_i|}{N}F(gs_i)
\end{equation}

The F-score measures both homogeneity (precision) and completeness (recall).
However, the F-score suffers from the \textit{matching problem}, by not
evaluating the entire membership of a cluster \citep{rosenberg2007v}. This is
due to the F-score not considering the components of the clusters beyond the
majority class. Furthermore, the F-score penalises systems for getting the
number of \ac{GS} classes wrongly \citep{manandhar2009semeval}.

Thus, to complement the F-score, the V-measure is also used. V-measure is an
entropy-based measure that explicitly measures how successfully the criteria
of homogeneity and completeness have been satisfied \citep{rosenberg2007v}. Just
as precision and recall are combined to form the F-score, homogeneity and
completeness are combined using the harmonic mean to compute the V-measure.

For the homogeneity criterion, a clustering must assign only the data points of a
single class to a single cluster. That is, the class distribution of each
cluster should be skewed to a single class, zero entropy \citep{rosenberg2007v}.
In a perfectly homogeneous case, $H(GS|C) = 0$ and in an imperfect situation,
this value is dependent on the size of the dataset and distribution of class
sizes. Therefore, the V-measure normalizes this value by the maximum reduction
in entropy the clustering could provide, $H(GS)$, resulting in
Equation~\ref{eq:homogeneity}.

\begin{equation} \label{eq:homogeneity}
 h = \begin{dcases}
      1,                        & \text{if } H(GS,C) = 0 \\
      1 - \frac{H(GS|C)}{H(GS)},& \text{otherwise} \\
     \end{dcases}
\end{equation}

\begin{equation}
 H(GS) = - \sum_{i=1}^{|GS|} \frac{\sum_{j=1}^{|C|} a_{ij}}{N}
         \log \frac{\sum_{j=1}^{|C|} a_{ij}}{N}
\end{equation}

\begin{equation}
 H(GS|C) = - \sum_{j=1}^{|C|} \sum_{i=1}^{|GS|} \frac{a_{ij}}{N}
           \log \frac{a_{ij}}{\sum_{k=1}^{|GS|} a_{kj}}
\end{equation}

Symmetrically to homogeneity, for the completeness criterion, a clustering
solution must assign all of the datapoints of a single class to a single
cluster. This can be evaluated by calculating the conditional entropy of the
proposed cluster distribution given the class of the component data points,
$H(C|GS)$. In a perfectly complete case, $H(C|GS) = 0$ and in the worst case
scenario each class is represented by every cluster with a distribution equal to
the distribution of cluster sizes, $H(C|GS)$ is maximal and equals $H(C)$.
Therefore, adding a way symmetric to that used for homogeneity, the V-measure
defines completeness as in Equation~\ref{eq:completeness}.

\begin{equation} \label{eq:completeness}
 c = \begin{dcases}
      1,                        & \text{if } H(C,GS) = 0 \\
      1 - \frac{H(C|GS)}{H(C)}, & \text{otherwise} \\
     \end{dcases}
\end{equation}

\begin{equation}
 H(C) = - \sum_{j=1}^{|C|} \frac{\sum_{i=1}^{|GS|} a_{ij}}{N}
        \log \frac{\sum_{i=1}^{|GS|} a_{ij}}{N}
\end{equation}

\begin{equation}
 H(C|GS) = - \sum_{i=1}^{|GS|} \sum_{j=1}^{|C|} \frac{a_{ij}}{N}
           \log \frac{a_{ij}}{\sum_{k=1}^{|C|} a_{ik}}
\end{equation}

Based on these calculations of homogeneity and completeness, the V-measure of a
clustering solution is then computed using the weighted harmonic mean of
homogeneity and completeness, according to Equation~\ref{eq:vmes}, in which if
$\beta$ is greater than 1 completeness is weighted more strongly and if less
than 1 homogeneity is weighted more strongly.

\begin{equation} \label{eq:vmes}
 V_\beta = \frac{(1+\beta)hc}{(\beta h) + c}
\end{equation}

Although the V-measure does not increase monotonically, it is known to tend to
favour systems producing a higher number of clusters than the number of \ac{GS}
senses \citep{manandhar2010semeval}. With this in mind, both the F-score and the
V-measure are used for this evaluation method, as the F-score penalises systems
when they produce a different number of clusters from the number of \ac{GS}
senses.

Additional measures for unsupervised evaluation include \textit{entropy} and
\textit{purity}. Entropy measures how various classes of objects are
distributed within each cluster. Generally, the smaller the entropy, the better
the clustering algorithm performs. On the other hand, Purity measures the extent
to which each cluster contains objects from primarily one class. The larger the
purity, the better the clustering algorithm performs. A formal definition of
these measures is available in \citep{zhao2005hierarchical}. However, as both of
them evaluate only the homogeneity of a clustering algorithm, disregarding
completeness \citep{manandhar2009semeval}, and are not frequently used in the
evaluation of more recent algorithms, they will not be considered in this
evaluation.

\section{Supervised Evaluation}
\label{sec:supeval}

In the supervised evaluation method, the target corpus is split into a testing
and training part. The training part is used to map the automatically induced
clusters to \ac{GS} senses \citep{agirre2006evaluating}. After that, the testing
corpus is used to evaluate the clustering resulting in a \ac{WSD} setting.

Suppose there are $m$ clusters and $n$ senses for the target word. $M$ is the
set of probabilities of words belonging to clusters, $M = \{m_{ij}\}$, $1 \leq i
\leq m, 1 \leq j \leq n$ and each $m_{ij} = P(gs_j|c_i)$, that is, $m_{ij}$ is
the probability of a word sense $j$ given it that has been assigned to a cluster
$i$. This probability can be computed counting the times an occurrence with
sense $j$ has been assigned to cluster $i$ in the train corpus.

The matrix $M$ is then used to transform any cluster score vector $\overline{h}$
returned by the algorithm into a sense vector $\overline{s}$. This is done by
multiplying the score vector by the matrix $M$, that is, $\overline{s} =
\overline{h}M$.

The mapping matrix $M$ is used in order to convert the cluster score vector
$\overline{h}$ of each test corpus instance into a sense score vector
$\overline{s}$, and then assign the sense with maximum score to that instance.

As the algorithm always returns an answer, its recall is of 100\% in all
cases, there are no false negatives as there are no negatives at all. So the
algorithm only needs to be evaluated according to its precision
\citep{agirre2006evaluating}.


% kate: default-dictionary en_GB; indent-width 2; replace-tabs on;
% kate: remove-trailing-space on; space-indent on;
% kate: replace-trailing-space-save on; remove-trailing-space on;


\section{Test Corpus}

To evaluate the project, the corpus from \citep{baptista2013viper} was used. 
This was a 290K word fragment of the PAROLE corpus \citep{nascimento1998parole}, with each verb instance manually annotated with its ViPEr class and reviewed by linguists. The splitting used for evaluation was to use the whole corpus for training, except for the sentence being evaluated.

\section{Parameter choosing}

The graph partitioning algorithm chosen was \ac{CW}, as the other option,
MaxMax, has a tendency to generate many fine-grained clusters
\citep{hope2013uos}.

The \ac{NPMI} and the logDice association measures were chosen as due to their
normalized scores, which allow to use the same parameter between different words
while keeping the same underlying meaning. The \ac{NPMI} association measure 
was tested with minimum thresholds of 0.0, 0.25, 0.5, and 0.75, ranging from 
each word being at best independent of the other up to both occurring mostly 
together. The logDice association measure was tested with minimum thresholds of 
0.0, 2.5, 5.0, 7.5, and 10.

\section{Unsupervised Evaluation}

The results of the unsupervised evaluation (Table~\ref{tab:unsup-results}) show 
that all tests scored poorly in F-Score, while \ac{MFS} had a result on average 
of 83.4\%. In the V-Measure, results varied from 0.0\% for the lowest values up 
to 34.4\% for the highest. Not counting the most restrictive threshold of 
logDice and NPMI, all tests had better results than \ac{MFS}, which scored 
0.4\% in V-Measure.

Another thing which is possible to see if that not enough points are available
to form a meaningful view of the contexts when the threshold is too high,
resulting in no clusters at all and giving poor results.

When evaluating the number of clusters, it is possible to see that most tests
might have been penalised due to the high number of clusters they had compared
to the average number of \ac{GS} senses.

\begin{table}[ht]
\caption{Results of the unsupervised \ac*{WSI} evaluation.}
\label{tab:unsup-results}
\begin{tabu} to \textwidth {Xlrrrr}
\hline
\textbf{Algorithm} & \textbf{Association Measure} & \textbf{Threshold} & \textbf{F-Score (\%)} & \textbf{V-Measure (\%)} & \textbf{\# Clusters} \\
\hline
\ac{CW} & logDice   &  0.0 &  1.95 & 33.62 & 147.3 \\
\ac{CW} & logDice   &  2.5 &  1.80 & 33.46 & 252.2 \\
\ac{CW} & logDice   &  5.0 &  2.46 & 29.60 & 259.7 \\
\ac{CW} & logDice   &  7.5 &  2.70 & 18.26 &  48.8 \\
\ac{CW} & logDice   & 10.0 &  0.83 &  3.33 &   0.1 \\
\hline
\ac{CW} & \ac{NPMI} & 0.0  &  2.34 & 30.97 &  76.5 \\
\ac{CW} & \ac{NPMI} & 0.25 &  1.69 & 34.37 & 380.1 \\
\ac{CW} & \ac{NPMI} & 0.5  &  0.72 &  9.80 &   0.2 \\
\ac{CW} & \ac{NPMI} & 0.75 &  0.00 &  0.00 &   0.0 \\
\hline
\ac{MFS} &      --- &  --- & 83.36 &  0.37 &   1.0 \\
\hline
\end{tabu}
\end{table}

\section{Supervised Evaluation}

The results on supervised \ac{WSD} (seen in Table~\ref{tab:sup-results}) were
very poor overall. None of the tests were able to surpass the results of
\ac{MFS}, with a precision of 65.7\%. The highest result was using logDice with
a threshold of 7.5, which reached a precision of 10.1\%.

\begin{table}[ht]
\caption{Results of the supervised \ac*{WSD} evaluation.}
\label{tab:sup-results}
\begin{tabu} to \textwidth {Xlrr}
\hline
\textbf{Algorithm} & \textbf{Association Measure} & \textbf{Threshold} & \textbf{Precision (\%)} \\
\hline
\ac{CW} & logDice   &  0.0 &  5.37 \\
\ac{CW} & logDice   &  2.5 &  0.00 \\
\ac{CW} & logDice   &  5.0 &  8.27 \\
\ac{CW} & logDice   &  7.5 & 10.10 \\
\ac{CW} & logDice   & 10.0 &  2.55 \\
\hline
\ac{CW} & \ac{NPMI} & 0.0  &  0.22 \\
\ac{CW} & \ac{NPMI} & 0.25 &  2.65 \\
\ac{CW} & \ac{NPMI} & 0.5  &  0.00 \\
\ac{CW} & \ac{NPMI} & 0.75 &  0.00 \\
\hline
\ac{MFS} & ---      &  --- & 65.74 \\
\hline
\end{tabu}
\end{table}

\section{Results interpretation and evaluation}

Overall, the tests had poor results. In all examples \ac{MFS} was able to
achieve better results, showing the project is not ready to be used for
disambiguation.

The high number of clusters obtained (on average above the hundreds) shows that
the results are too fine-grained to be able to properly match them to the 
senses one is trying to disambiguate.

Further inspection into specific graphs of some words, such as the graph for the
word \emph{vingar} (Figure~\ref{fig:vingar_graph}) can further explain the
obtained results.

\begin{figure}[ht]
  \caption{Image of the induction graph for the word \emph{vingar}, using the
    \ac*{CW} algorithm and the \ac*{NPMI} association measure.}
  \label{fig:vingar_graph}
  \centering
  \includegraphics[width=\textwidth]{graphics/vingar_VERB_npmi_0_4_chineseWhispers}
\end{figure}

The first noticeable thing is that the graph includes a few words with spelling
errors, such as the nodes \emph{natambém} and \emph{estratrégia}. The second 
noticeable thing is that although the work by \citet{correia2015syntax} 
identifies named entities and replaces their lemmas with their categories, many 
of the words in the graph are named entities which were not recognized as such 
by \ac{STRING}. This can be seen in nodes such as \emph{Windsor} and 
\emph{Shrek}, and adds noise to the graph, increasing the number of small 
clusters generated.

But the most noticeable thing is how sparse the graph is. Algorithms such as
\ac{CW} or MaxMax require a \emph{small-world network} with several high-density
areas to be able to find clusters in the graph. In a graph such as the one
in Figure~\ref{fig:vingar_graph}, with the exception of the node corresponding
to the target word, no nodes have more than 2 neighbours. This undermines the
assumptions used in graph-clustering algorithms, and prevents the possibility
of better results.

It is possible the graph is sparse because the syntactic dependencies used
impose a stricter relationship between the two words than the usage of mere
words which co-occur in the same sentence or paragraph would. The stricter
relationship between the words changes the behaviour of the resulting graph,
which make the graph-clustering algorithms behave poorly.

Additionally, the stricter relationship might be preventing words that are
related but do not have a syntactic relationship from being included in the
graph. This might make the generated graphs unsuitable for the specific \ac{WSI}
algorithms used in this project.

Another possible cause of the poor results might be the absence of categories, 
such as \texttt{PERSON}, \texttt{PLACE} or \texttt{ORG}, among others, in the 
algorithm used. As it is blind to the categories, the algorithm can not make 
use of them to help infer the senses of the target word.

% kate: default-dictionary en_GB; indent-width 2; replace-tabs on;
% kate: remove-trailing-space on; space-indent on;
% kate: replace-trailing-space-save on; remove-trailing-space on;
