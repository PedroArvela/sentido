\chapter{Evaluation}

\section{Test Corpus}

To evaluate the project, the corpus from \cite{pires2017verb} was used. This was
the PAROLE corpus \cite{nascimento1998parole}, with each verb manually annotated
with its ViPEr class and reviewed by linguists.

The splitting used for evaluation was to use the whole corpus for training,
except for the sentence being evaluated.

\section{Parameter choosing}

The graph partitioning algorithm chosen was \ac{CW}, as the other option,
MaxMax, has a tendency to generate many fine-grained clusters
\cite{hope2013uos}.

\ac{NPMI} and logDice were chosen as the association measures due to their
normalized scores, which allow to use the same parameter between different words
while keeping the same underlying meaning.

\ac{NPMI} was tested with minimum thresholds of 0.0, 0.25, 0.5, and 0.75,
ranging from each word being at best independent of the other up to both
occurring mostly together.

logDice was tested with minimum thresholds of 0.0, 2.5, 5.0, 7.5, and 10,
ranging from being at least 1 co-occurrence per 16,000 instances of each
individual word, up to 1 co-occurrence per 15.625 instances of each word.

\section{Unsupervised Evaluation}

The results of the unsupervised evaluation (shown in
Table~\ref{tab:unsup-results}), shows that all tests scored poorly in F-Score,
while \ac{MFS} had a result on average of 83.4\%. In the V-Measure, results
varied from 0.000 for the lowest values up to 0.344 to the highest. Not counting
the most restrictive threshold of logDice and NPMI, all tests had better results
than \ac{MFS}, which scored 0.004 in V-Measure.

Another thing which is possible to see if that when the threshold is too high,
not enough points are available to form a meaningful view of the system,
resulting in no clusters at all and giving poor results.

When evaluating the number of clusters, it is possible to see that most tests
might have been penalised due to the high number of clusters they had compared
in relation to the average number of gold senses.

\begin{table}[ht]
\caption{Results of the unsupervised \ac{WSI} evaluation. All results on logDice
and \ac{NPMI} were made using \ac{CW}.}
\label{tab:unsup-results}
\begin{tabu} to \textwidth {Xrrrr}
\hline
\textbf{Association Measure} & \textbf{Threshold} & \textbf{F-Score (\%)} & \textbf{V-Measure} & \textbf{\# Clusters} \\
\hline
logDice   &  0.0 & 0.0195 & 0.3362 & 147.3 \\
logDice   &  2.5 & 0.0180 & 0.3346 & 252.2 \\
logDice   &  5.0 & 0.0246 & 0.2960 & 259.7 \\
logDice   &  7.5 & 0.0270 & 0.1826 &  48.8 \\
logDice   & 10.0 & 0.0083 & 0.0333 &   0.1 \\
\hline
\ac{NPMI} & 0.0  & 0.0234 & 0.3097 &  76.5 \\
\ac{NPMI} & 0.25 & 0.0169 & 0.3437 & 380.1 \\
\ac{NPMI} & 0.5  & 0.0072 & 0.0980 &   0.2 \\
\ac{NPMI} & 0.75 & 0.0000 & 0.0000 &   0.0 \\
\hline
\ac{MFS}  &  --- & 0.8336 & 0.0037 &   1.0 \\
\hline
\end{tabu}
\end{table}


% kate: default-dictionary en_GB; indent-width 2; replace-tabs on;
% kate: remove-trailing-space on; space-indent on;
% kate: replace-trailing-space-save on; remove-trailing-space on;
