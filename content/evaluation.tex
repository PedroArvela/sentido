\chapter{Evaluation}
\label{ch:eval}

\section{Test Corpus}

To evaluate the project, the corpus from \citep{pires2017verb} was used. This 
was
the PAROLE corpus \citep{nascimento1998parole}, with each verb manually 
annotated
with its ViPEr class and reviewed by linguists.

The splitting used for evaluation was to use the whole corpus for training,
except for the sentence being evaluated.

\section{Parameter choosing}

The graph partitioning algorithm chosen was \ac{CW}, as the other option,
MaxMax, has a tendency to generate many fine-grained clusters
\citep{hope2013uos}.

\ac{NPMI} and logDice were chosen as the association measures due to their
normalized scores, which allow to use the same parameter between different words
while keeping the same underlying meaning.

\ac{NPMI} was tested with minimum thresholds of 0.0, 0.25, 0.5, and 0.75,
ranging from each word being at best independent of the other up to both
occurring mostly together.

logDice was tested with minimum thresholds of 0.0, 2.5, 5.0, 7.5, and 10,
ranging from being at least 1 co-occurrence per 16,000 instances of each
individual word, up to 1 co-occurrence per 15.625 instances of each word.

\section{Unsupervised Evaluation}

The results of the unsupervised evaluation (shown in
Table~\ref{tab:unsup-results}), shows that all tests scored poorly in F-Score,
while \ac{MFS} had a result on average of 83.4\%. In the V-Measure, results
varied from 0.0\% for the lowest values up to 34.4\% for the highest. Not
counting the most restrictive threshold of logDice and NPMI, all tests had
better results than \ac{MFS}, which scored 0.4\% in V-Measure.

Another thing which is possible to see if that when the threshold is too high,
not enough points are available to form a meaningful view of the contexts,
resulting in no clusters at all and giving poor results.

When evaluating the number of clusters, it is possible to see that most tests
might have been penalised due to the high number of clusters they had compared
in relation to the average number of gold senses.

\begin{table}[ht]
\caption{Results of the unsupervised \ac*{WSI} evaluation.}
\label{tab:unsup-results}
\begin{tabu} to \textwidth {Xlrrrr}
\hline
\textbf{Algorithm} & \textbf{Association Measure} & \textbf{Threshold} & \textbf{F-Score (\%)} & \textbf{V-Measure (\%)} & \textbf{\# Clusters} \\
\hline
\ac{CW} & logDice   &  0.0 &  1.95 & 33.62 & 147.3 \\
\ac{CW} & logDice   &  2.5 &  1.80 & 33.46 & 252.2 \\
\ac{CW} & logDice   &  5.0 &  2.46 & 29.60 & 259.7 \\
\ac{CW} & logDice   &  7.5 &  2.70 & 18.26 &  48.8 \\
\ac{CW} & logDice   & 10.0 &  0.83 &  3.33 &   0.1 \\
\hline
\ac{CW} & \ac{NPMI} & 0.0  &  2.34 & 30.97 &  76.5 \\
\ac{CW} & \ac{NPMI} & 0.25 &  1.69 & 34.37 & 380.1 \\
\ac{CW} & \ac{NPMI} & 0.5  &  0.72 &  9.80 &   0.2 \\
\ac{CW} & \ac{NPMI} & 0.75 &  0.00 &  0.00 &   0.0 \\
\hline
\ac{MFS} &      --- &  --- & 83.36 &  0.37 &   1.0 \\
\hline
\end{tabu}
\end{table}

\section{Supervised Evaluation}

The results on supervised \ac{WSD} (seen in Table~\ref{tab:sup-results}) were
very poor overall. None of the tests were able to surpass the results of
\ac{MFS}, with a precision of 65.7\%. The highest result was using logDice with
a threshold of 7.5, which reached a precision of 10.1\%.

\begin{table}[ht]
\caption{Results of the supervised \ac*{WSD} evaluation.}
\label{tab:sup-results}
\begin{tabu} to \textwidth {Xlrr}
\hline
\textbf{Algorithm} & \textbf{Association Measure} & \textbf{Threshold} & \textbf{Precision (\%)} \\
\hline
\ac{CW} & logDice   &  0.0 &  5.37 \\
\ac{CW} & logDice   &  2.5 &  0.00 \\
\ac{CW} & logDice   &  5.0 &  8.27 \\
\ac{CW} & logDice   &  7.5 & 10.10 \\
\ac{CW} & logDice   & 10.0 &  2.55 \\
\hline
\ac{CW} & \ac{NPMI} & 0.0  &  0.22 \\
\ac{CW} & \ac{NPMI} & 0.25 &  2.65 \\
\ac{CW} & \ac{NPMI} & 0.5  &  0.00 \\
\ac{CW} & \ac{NPMI} & 0.75 &  0.00 \\
\hline
\ac{MFS} & ---      &  --- & 65.74 \\
\hline
\end{tabu}
\end{table}

\section{Result interpretation and evaluation}

Overall, the tests had poor results. In all examples \ac{MFS} was able to
achieve better results, showing the project is not ready to be used for
disambiguation.

The high number of clusters obtained (on average above the hundreds) shows that
the results are too fine-grained to be able to properly match them to the senses
trying to disambiguate.

This might be caused by the use of syntactic dependencies, which give many less
co-occurrences, ad thus not being able to make use of the small word properties
of co-occurrence graphs described in Section~\ref{sec:co-occurrence}.

Another relevant point which might affect the results is that the extractor used
\citep{correia2015syntax} never had in mind the use case of this project, and 
was
filtering several co-occurrences, as seen in Figure~\ref{fig:extractor}. It is
possible and likely that it could have filtered co-occurrences essential to be
able to disambiguate the use of a word in a given context. The co-occurrences
it was filtering might also have affected the small word properties of the
co-occurrence graph, invalidating the foundations in which the graph clustering
algorithms work on to be able to achieve acceptable results.

\begin{figure}[ht]
\caption[Dependencies found by \acs*{STRING} and the extractor]{A comparison between
the dependencies found by STRING and the ones used by the extractor. All the
dependencies shown were found by \ac{STRING}, while only the ones shown in blue
were recognized by the extractor.}
\label{fig:extractor}
\centering
\tikzstyle{every node}=[text height=1.5ex, text depth=0.25ex,
                        minimum width=1cm, text centered]
\tikzstyle{every path} = [draw=gray, -triangle 45]
\begin{tikzpicture}[node distance=1.55cm, on grid]
\node (1) {O};
\node [right of=1] (2) {Pedro};
\node [right of=2] (3) {vingou};
\node [right of=3] (4) {se};
\node [right of=4] (5) {de};
\node [right of=5] (6) {quem};
\node [right of=6] (7) {matou};
\node [right of=7] (8) {o};
\node [right of=8] (9) {seu};
\node [right of=9] (10) {pai};

\path (2) -- (1);
\path (10) to[bend right] (8);
\path (6) -- (5);
\path (7) to[bend right] (3);
\path (10) -- (9);
\path [blue] (3) -- (2);
\path [blue] (7) to[bend right] (2);
\path (4) -- (3);
\path [blue] (7) to[bend left] (10);
\path (6) -- (7);
\end{tikzpicture}
\end{figure}

Additionally, even though \ac{CW} produces coarser clusters than MaxMax, its
results are still fine-grained and produce more clusters than related algorithms
\citep{marco2013clustering}. Using a different algorithm, such as \ac{B-MST}
\citep{marco2013clustering} or HyperLex \citep{veronis2004hyperlex} could 
possibly
have achieved better results.

% kate: default-dictionary en_GB; indent-width 2; replace-tabs on;
% kate: remove-trailing-space on; space-indent on;
% kate: replace-trailing-space-save on; remove-trailing-space on;
