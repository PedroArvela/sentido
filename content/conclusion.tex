\chapter{Conclusion}
\label{ch:conclusion}

This project proposed an algorithm for disambiguating the sense of words in the
Portuguese language without the need to manually create a sense inventory or
sense-tagged corpora.

Additionally, this project implemented a proof-of-concept of this same proposal,
and evaluated its results against a \ac{MFS} baseline.

Although the project was unable to achieve satisfactory results in the
evaluation, it was possible to pinpoint were possible weaknesses were and a base
model was defined and documented, which can be used for future attempts at
solving the given problem.

At last, this project shows in its State of the Art extensive examples of
interest and successful attempts, showing that not only is this goal beneficial
for \ac{NLP} tasks, as well as possible to do, even if this specific
implementation was not capable of achieving good results.

One important aspect to investigate is the possibility of adding more
dependency types to the ones detected by Syntax Deep Explorer
\citep{correia2015syntax}. Not only would these help improve the results
obtained
by \ac{SENTIDO} as they would also be useful for Syntax Deep Explorer's original
goal.

Another aspect would be to investigate further clustering algorithms, such as
\ac{B-MST} \citep{marco2013clustering} or HyperLex \citep{veronis2004hyperlex},
as
they could help improve the results of both the induction and the disambiguation
phase.

Another point worth investigating is further processing before or after
generating the graph or during induction, such as choosing different association
measures and thresholds, or including information about the dependency patterns
themselves in the generated clusters for when applying the disambiguation
\citep{panchenko2016noun}.

Additionally, it might be relevant to investigate porting the data to a graph
database engine, as it's native work in graphs might improve the execution speed
of the induction phase by removing several of the currently required
\texttt{JOIN}s of multiple tables in relational databases.

% kate: default-dictionary en_GB; indent-width 2; replace-tabs on;
% kate: remove-trailing-space on; space-indent on;
% kate: replace-trailing-space-save on; remove-trailing-space on;
