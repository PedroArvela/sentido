\chapter{Conclusion}

This project proposed an algorithm for disambiguating the sense of words in the
Portuguese language without the need to manually create a sense inventory or
sense-tagged corpora.

Additionally, this project implemented a proof-of-concept of this same proposal,
and evaluated its results against a \ac{MFS} baseline.

Although the project was unable to achieve satisfactory results in the
evaluation, it was possible to pinpoint were possible weaknesses were and a base
model was defined and documented, which can be used for future attempts at
solving the given problem.

At last, this project shows in its State of the Art extensive examples of
interest and successful attempts, showing that not only is this goal beneficial
for \ac{NLP} tasks, as well as possible to do, even if this specific
implementation was not capable of achieving good results.

% kate: default-dictionary en_GB; indent-width 2; replace-tabs on;
% kate: remove-trailing-space on; space-indent on;
% kate: replace-trailing-space-save on; remove-trailing-space on;
