\chapter{Conclusion}
\label{ch:conclusion}

This project proposed an algorithm for disambiguating the sense of words in the
Portuguese language without the need to manually create a sense inventory or
sense-tagged corpora. Additionally, this project implemented a proof-of-concept
of this same proposal, and evaluated its results against a \ac{MFS} baseline.

One future investigation point would be to evaluate the use of relaxed
co-occurrences based on context windows, sentences or even paragraphs, instead
of the syntactic dependency co-occurrences used in this project. This might
influence the characteristics of the generated graphs and improve the
performance of the graph-clustering algorithms.

Another aspect should be to investigate vector-based algorithms. These might be
capable of making use of the dependency information provided by \ac{XIP}, and
might perform better in relation to algorithms blind to this
information, such as the ones used in this project.

Additionally, it might be relevant to investigate porting the data to a graph
database engine. The traditional relational database used requires the use of
multiple \texttt{JOIN}s to execute the required queries. A database with a
native concept of a graph would allow to remove these \texttt{JOIN}s and improve
execution speed.

Although the project was unable to achieve satisfactory results in the
evaluation, it was possible to pinpoint where possible weaknesses are located.
A base model was defined and documented, and it can be used as a base for future
attempts at solving the challenges found.

% kate: default-dictionary en_GB; indent-width 2; replace-tabs on;
% kate: remove-trailing-space on; space-indent on;
% kate: replace-trailing-space-save on; remove-trailing-space on;
