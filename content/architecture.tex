\chapter{Architecture}
\label{ch:architecture}

The architecture of this project is composed of four components:

\begin{enumerate}
  \item A corpora pre-parser, which prepares the corpora being used to be
processed by \ac{STRING};
  \item The co-occurrence extractor from \cite{correia2015syntax};
  \item A graph constructor and clustering algorithm;
  \item A sense disambiguation module.
\end{enumerate}

The data follows the flow exemplified in Figure~\ref{fig:data-progression}. The
processed text from \ac{STRING} goes through a modified implementation of the
co-occurrence extractor from \cite{correia2015syntax}, which stores the
co-occurrences along with their frequencies and association measures in a
database.

\begin{figure}[ht]
 \caption{The progression of data as it evolves through the architecture}
 \label{fig:data-progression}
 \centering
 \tikzstyle{block} = [rectangle, draw,
    text width=6em, text centered, minimum height=3em]
\tikzstyle{line} = [draw, -latex']

\begin{tikzpicture}[node distance = 2cm, on grid]
    % Place nodes
    \node [block] (init) {POS-tagged paragraphs};
    \node [block, below of=init] (pairs) {word pairs};
    \node [block, below left of=pairs, node distance=3cm] (nodes) {nodes};
    \node [block, below right of=pairs, node distance=3cm] (edges) {edges};
    \node [block, below right of=nodes, node distance=3cm] (cooc) {co-occurrence graph};
    \node [block, below of=cooc, node distance=2.5cm] (cluster) {cluster of nodes};

    \path [line] (init) -- (pairs) node [midway, right] {co-occurrence extraction};
    \path [line] (pairs) -| (nodes) node [near end, left] {words};
    \path [line] (pairs) -| (edges) node [near end, right] {relations};
    \path [line] (nodes) |- (cooc);
    \path [line] (edges) |- (cooc);
    \path [line] (cooc) -- (cluster) node [near end, right] {clustering algorithm};

    \draw ($(nodes.north west) + (-0.5,0.5)$) rectangle ($(cooc.south east) + (2.6,-0.5)$);
    \node [below left of=cooc, xshift=-1cm] () {graph construction};
\end{tikzpicture}

\end{figure}

\section{Co-occurrence Storage}

The co-occurrences are stored in an SQLite database, using the \ac{ER} model in
Figure~\ref{fig:er-model}, adapted from \cite{correia2015syntax}, with minor
changes to keep information of all the sentences used instead of only 20
randomly selected sentences for each co-occurrence.

\begin{figure}[ht]
  \caption{The \acl*{ER} model used to store the information in the database}
  \label{fig:er-model}
  \centering
  \tikzstyle{every node}=[font=\small]
%\tikzstyle{every entity}=[]
\tikzstyle{every attribute}=[font=\small]

\tikzstyle{weak entity}=[entity, line width=0.8pt, double, double distance=1pt]
\tikzstyle{weak relationship}=[relationship, double, double distance=1pt]

\begin{tikzpicture}[node distance = 2cm, on grid]
  \node [entity] (corpus) at (0,0) {Corpus}
    child {node [key attribute] (cname) at (1,1.5) {\underline{name}}}
    child {node [attribute] (csource) at (-1,3) {source}}
    child {node [attribute] (cyear) at (-1,3) {year}}
    child {node [attribute] (cgenre) at (-1,3) {genre}}
    child {node [attribute] (cupdate) at (-1,3) {update}};

  \node [weak entity] (file) at (4,0) {File}
    child {node [key attribute] (fname) at (0,3) {\dashuline{name}}};

  \node [entity] (word) at (0,-5) {Word}
    child {node [key attribute] (wword) at (-1,2) {\underline{word}}}
    child {node [key attribute] (wclass) at (-2.5,1) {\underline{class}}};

  \node [weak entity] (context) at (8,0) {Context}
    child {node [key attribute] (csentence) at (0,3) {\dashuline{sentence}}};

  \node [entity] (dependency) at (8,-1.5) {Dependency}
    child {node [key attribute] (dtype) at (2,1.5) {\underline{type}}};

  \node [weak entity] (property) at (8,-5) {Property}
    child {node [key attribute] (ptype) at (2,1.5) {\dashuline{type}}};

  \node [relationship] (co-occurrence) at (4,-5) {co-occurrence}
    edge [-latex] (word)
    edge [-latex] (property)
    edge [-latex] (corpus)
    child {node [attribute] {frequency}}
    child {node [attribute] {pmi}}
    child {node [attribute] {npmi}}
    child {node [attribute] {dice}}
    child {node [attribute] {logDice}}
    child {node [attribute] at (0,-0.5) {chiPearson}}
    child {node [attribute] {logLikelihood}}
    child {node [attribute] at (0,-0.5){significance}};

  \draw [-latex] (co-occurrence) -- (2,-4.5) -- (word);

  \node [weak relationship] at (8,-2.8) {has}
    edge [-latex] (dependency)
    edge [line width=0.8pt, double, double distance=1pt] (property);

  \node [relationship] at (0,-2) {belongs}
    edge [-latex] (corpus)
    edge [-latex] (property)
    edge [-latex] (word)
    child {node [attribute] (bfreq) at (-2.5,1.5) {frequency}};

  \node [weak relationship] at (2,0) {has}
    edge [-latex] (corpus)
    edge [line width=0.8pt, double, double distance=1pt] (file);

  \node [weak relationship] at (6,0) {has}
    edge [-latex] (file)
    edge [line width=0.8pt, double, double distance=1pt] (context);

  \node [draw,dashed, minimum width=15cm, minimum height=3.6cm] (agg) at ([yshift=-0.5cm]co-occurrence) {};

  \node [relationship] at (4,-2) {occurs}
    edge [-latex] (agg)
    edge [-latex] (context)
    child {node [attribute] at (2, 1) {frequency}};
\end{tikzpicture}

\end{figure}

The changes in relation to \cite{correia2015syntax} are that the entity
\emph{Context} replaces \emph{Sentence}, with information about the file used to
store the information as well as the sentence number in that file.

Additionally, the relationship \emph{Exemplifies} is replaced with
\emph{Occurs}, which associates all \emph{Co-occurrence} aggregations with all
the respective \emph{Context}s where they occur.

As in \cite{correia2015syntax}, the following \ac{IC} were identified:

\begin{enumerate}
  \item The word in \emph{Co-occurrence} must belong to the \emph{Corpus} to
    which they were associated with;
  \item The \emph{Co-occurrence} association must be associated to the same
    \emph{Property} to which the words associate with in \emph{Belongs};
  \item The sentences in \emph{Context} must belong to the same \emph{Corpus}
    as the \emph{Co-occurrences} which occur in these.
\end{enumerate}

\section{Graph Generation and Clustering}

For a target word $w$, a query to the database is made to obtain all
co-occurrences which occur in the same contexts as $w$, along with the
respective association measures. After filtering all co-occurrences which do not
reach the minimum threshold value of the association measure being used, the
co-occurrences are saved in a graph structure.

To ensure only words directly related remain, a breadth-first search is made
starting from the target word. Only the nodes which were visited during this
process are kept in the final graph.

After the graph is generated, a graph clustering algorithm is ran against it,
and the resulting senses are stored.

\section{Sense Disambiguation}

To disambiguate senses of a target word $w$ from a given context
$c$, the co-occurrences from $c$ are extracted and used to generate
the co-occurrence cluster for the context, $C_i$.

Then, for each inferred sense cluster $C_j$ of $w$, the
\emph{Separation} between $C_i$ and $C_j$ is calculated according to
Equation~\ref{eq:separation} \cite{hope2013uos}, in which
$\operatorname{proximity}$ is defined as the weight of the co-occurrence in the
inferred sense graph.

\begin{equation}
 \operatorname{separation}(C_i,C_j) =
 1 - \left(
 \frac {\sum_{\substack{x \in C_i \\ y \in C_j}} \operatorname{proximity}(x,y)}
       {|C_i| \times |C_j|}
 \right)
\label{eq:separation}
\end{equation}

The cluster $C_j$ with the lowest separation score compared to $C_i$ is then
considered the most likely sense of the target word $w$.


% kate: default-dictionary en_GB; indent-width 2; replace-tabs on;
% kate: remove-trailing-space on; space-indent on;
% kate: replace-trailing-space-save on; remove-trailing-space on;
