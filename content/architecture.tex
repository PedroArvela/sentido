\chapter{Architecture}
\label{sec:architecture}

The architecture of this project is composed of four components:

\begin{enumerate}
  \item A corpora pre-parser, which prepares the corpora being used to be
processed by \ac{STRING};
  \item The co-occurrence extractor from \cite{correia2015syntax};
  \item A graph constructor and clustering algorithm;
  \item A sense disambiguation module.
\end{enumerate}

The data follows the flow exemplified in Figure~\ref{fig:data-progression}. The
processed text from \ac{STRING} goes through a modified implementation of the
co-occurrence extractor from \cite{correia2015syntax}, which stores the
co-occurrences along with their frequencies and association measures in a
database.

\begin{figure}[h]
 \centering
 \tikzstyle{block} = [rectangle, draw,
    text width=6em, text centered, minimum height=3em]
\tikzstyle{line} = [draw, -latex']

\begin{tikzpicture}[node distance = 2cm, on grid]
    % Place nodes
    \node [block] (init) {POS-tagged paragraphs};
    \node [block, below of=init] (pairs) {word pairs};
    \node [block, below left of=pairs, node distance=3cm] (nodes) {nodes};
    \node [block, below right of=pairs, node distance=3cm] (edges) {edges};
    \node [block, below right of=nodes, node distance=3cm] (cooc) {co-occurrence graph};
    \node [block, below of=cooc, node distance=2.5cm] (cluster) {cluster of nodes};

    \path [line] (init) -- (pairs) node [midway, right] {co-occurrence extraction};
    \path [line] (pairs) -| (nodes) node [near end, left] {words};
    \path [line] (pairs) -| (edges) node [near end, right] {relations};
    \path [line] (nodes) |- (cooc);
    \path [line] (edges) |- (cooc);
    \path [line] (cooc) -- (cluster) node [near end, right] {clustering algorithm};

    \draw ($(nodes.north west) + (-0.5,0.5)$) rectangle ($(cooc.south east) + (2.6,-0.5)$);
    \node [below left of=cooc, xshift=-1cm] () {graph construction};
\end{tikzpicture}

 \caption{The progression of data as it evolves through the architecture}
 \label{fig:data-progression}
\end{figure}

\section{Co-occurrence and Graph Storage}

\begin{figure}[h]
  \centering
  \tikzstyle{every node}=[font=\small]
%\tikzstyle{every entity}=[]
\tikzstyle{every attribute}=[font=\small]

\tikzstyle{weak entity}=[entity, line width=0.8pt, double, double distance=1pt]
\tikzstyle{weak relationship}=[relationship, double, double distance=1pt]

\begin{tikzpicture}[node distance = 2cm, on grid]
  \node [entity] (corpus) at (0,0) {Corpus}
    child {node [key attribute] (cname) at (1,1.5) {\underline{name}}}
    child {node [attribute] (csource) at (-1,3) {source}}
    child {node [attribute] (cyear) at (-1,3) {year}}
    child {node [attribute] (cgenre) at (-1,3) {genre}}
    child {node [attribute] (cupdate) at (-1,3) {update}};

  \node [weak entity] (file) at (4,0) {File}
    child {node [key attribute] (fname) at (0,3) {\dashuline{name}}};

  \node [entity] (word) at (0,-5) {Word}
    child {node [key attribute] (wword) at (-1,2) {\underline{word}}}
    child {node [key attribute] (wclass) at (-2.5,1) {\underline{class}}};

  \node [weak entity] (context) at (8,0) {Context}
    child {node [key attribute] (csentence) at (0,3) {\dashuline{sentence}}};

  \node [entity] (dependency) at (8,-1.5) {Dependency}
    child {node [key attribute] (dtype) at (2,1.5) {\underline{type}}};

  \node [weak entity] (property) at (8,-5) {Property}
    child {node [key attribute] (ptype) at (2,1.5) {\dashuline{type}}};

  \node [relationship] (co-occurrence) at (4,-5) {co-occurrence}
    edge [-latex] (word)
    edge [-latex] (property)
    edge [-latex] (corpus)
    child {node [attribute] {frequency}}
    child {node [attribute] {pmi}}
    child {node [attribute] {npmi}}
    child {node [attribute] {dice}}
    child {node [attribute] {logDice}}
    child {node [attribute] at (0,-0.5) {chiPearson}}
    child {node [attribute] {logLikelihood}}
    child {node [attribute] at (0,-0.5){significance}};

  \draw [-latex] (co-occurrence) -- (2,-4.5) -- (word);

  \node [weak relationship] at (8,-2.8) {has}
    edge [-latex] (dependency)
    edge [line width=0.8pt, double, double distance=1pt] (property);

  \node [relationship] at (0,-2) {belongs}
    edge [-latex] (corpus)
    edge [-latex] (property)
    edge [-latex] (word)
    child {node [attribute] (bfreq) at (-2.5,1.5) {frequency}};

  \node [weak relationship] at (2,0) {has}
    edge [-latex] (corpus)
    edge [line width=0.8pt, double, double distance=1pt] (file);

  \node [weak relationship] at (6,0) {has}
    edge [-latex] (file)
    edge [line width=0.8pt, double, double distance=1pt] (context);

  \node [draw,dashed, minimum width=15cm, minimum height=3.6cm] (agg) at ([yshift=-0.5cm]co-occurrence) {};

  \node [relationship] at (4,-2) {occurs}
    edge [-latex] (agg)
    edge [-latex] (context)
    child {node [attribute] at (2, 1) {frequency}};
\end{tikzpicture}

  \caption{The \acl*{ER} model used to store the information in the database}
  \label{fig:er-model}
\end{figure}


% kate: default-dictionary en_GB; indent-width 2; replace-tabs on;
% kate: remove-trailing-space on; space-indent on;
% kate: replace-trailing-space-save on; remove-trailing-space on;
