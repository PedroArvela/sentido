\chapter{Introduction}

\section{Motivation}

In most human languages, it is possible that the same word has different
meanings according to the context in which it is used. For example:

\begin{enumerate}[label=(\alph*)]
 \item She shot the arrow using her \textit{bow}.
 \item He tied the \textit{bow} of the gift.
\end{enumerate}

In these two sentences, the word \textit{bow} means different things. In
sentence \textit{(a)} it is used as a hunting tool, while in sentence
\textit{(b)} it is a ribbon. To be able to do some tasks in \ac{NLP}, machines
need to be able to differentiate between each use of the same word.

\ac{WSD} is thus the identification of the underlying meaning of words in their
context \cite{navigli2009word}.

To be able to accomplish tasks of \ac{WSD}, it is necessary to have information
about the various words, their possible meanings, and how each meaning is
usually expressed.

When there is a lack of lexical resources, such as sense inventories or
sense-tagged corpora, preforming \ac{WSD} can be a time-consuming task which
needs to be repeated any time there is a change in the domain, language or
sense inventory in use \cite{ng1997getting}.

This problem becomes aggravated by the exponential growth of data published on
the internet \cite{james2014data,james2016data} and the accompanying need to
process this data through automatic methods.

Additionally, \ac{WSD} preforms a crucial role in machine translation. The
Portuguese word for \emph{banco} in English can either mean \emph{bank},
\emph{chair}, \emph{stool}, among other meanings, depending upon context. There
are several cases where \ac{WSD} can have large influence in the quality of
automatically translated text \cite{navigli2009word}.

\ac{WSI} is thus the task of automatically identifying the meanings of words
without presence of manually annotated data \cite{agirre2007semeval}. Many
systems have been implemented to preform the task of \ac{WSI}, some of which
will be later described in Section~\ref{ch:stateofart}.

\section{Objectives}

\ac{SENTIDO}'s goal is to create an architecture capable of generating
labels for a word which are related to the word's semantic meanings, using the
context in a corpus to preform such task. Additionally, it intends to provide a
mechanism, given a word, its induced labels, and a context, to disambiguate
which label most closely corresponds to the usage of the word in this context.


% kate: default-dictionary en_GB; indent-width 2; replace-tabs on;
% kate: remove-trailing-space on; space-indent on;
% kate: replace-trailing-space-save on; remove-trailing-space on;
