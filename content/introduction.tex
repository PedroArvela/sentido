\chapter{Introduction}

% Explain the problem

In human languages, it is possible that the same word has different meanings
according to the context in which it is used. For example:

\begin{enumerate}[label=(\alph*)]
 \item She shot the arrow using her \textit{bow}.
 \item He tied the \textit{bow} of the gift.
\end{enumerate}

In these two sentences, the word \textit{bow} means different things. In
sentence (a) it is used as a hunting tool, while in sentence (b) it is a ribbon.
To be able to do some tasks in \ac{NLP}, machines need to be able to
differentiate between each use of the same word.

% Show why the problem is ubiquitous

This problem affects machine translation, \ac{IR} and content categorization
\cite{navigli2009word}. Taking an example of machine translation, the
Portuguese word \textit{laço} can be translated in various different English
words depending on context; it can be translated into \textit{ribbon} or it can
be translated to the \textit{bond} between two people. A system which naïvely
uses the \ac{MFS} will improperly translate or categorize these words when they
are used outside of their most common context.

% Introduce the solution to the problem

To deal with this problem it is possible to identify the underlying meaning of a
word based on its context. This process is called \ac{WSD}
\cite{navigli2009word}. This process requires both sense inventories as well as
large amounts of sense-tagged corpora to function efficiently. As a result,
under-resourced languages need to deal with greater hardships to be able to
achieve satisfactory results \cite{ng1997getting}.

A solution to the lack of resources is to automatically identify the meaning of
words in their given context, without the requirement of manually annotated
data. This is called \ac{WSI} \cite{agirre2007semeval}.

% Propose questions which this paper will answer

% Is it possible?
% Does it execute within reasonable limits?
% Can it be better than no induction/disambiguation?
% How do you evaluate it? Aka, how do you know it is better?

There are several implementations of \ac{WSI} available (some of which will be
later described in Chapter~\ref{ch:stateofart}). The goal of this dissertation
is to investigate the feasibility and possibility of creating a \ac{WSI} model
for the Portuguese language which is capable of improving the quality of
detected word senses within the words' contexts. Furthermore, this dissertation
looks into seeing if the task of \ac{WSI} and \ac{WSD} can be preformed within
reasonable time and resources, and to evaluate the quality of this new model
against the existing baselines used within the \ac{NLP} chain \acs*{STRING}
(described in Section~\ref{sec:string}).

% Introduce the shiny new tool, explain the problem it solves

Additionally, this dissertation presents the results of this investigation, in
a project called \ac{SENTIDO}. \ac{SENTIDO} is a \ac{WSI} and \ac{WSD} model
which infers the possible senses of a word from untagged corpora based on the
additional words which co-occur with the target word; and, given a word and its
context, disambiguates between the previously inferred senses.

% Describe the layout of this paper

In Chapter~\ref{ch:stateofart}, the existing \ac{WSI} implementations and models
are described, as well and the theoretical foundations and additional tools
which support and aide them. In Chapter~\ref{ch:architecture}, the architecture
of the model is outlined, as well as the various stages of which it is composed
of. In Chapter~\ref{ch:implementation}, the implementation details, such as the
used corpora, existing tools from which the model is developed on, and
adaptations made to those same tools, are depicted.
Chapter~\ref{ch:eval-method} describes how the algorithm is tested and which are
the theoretical foundations for the soundness of the methodology.
Chapter~\ref{ch:eval} describes the test corpus used, the chosen parameters and
analyses the results of the evaluation. Chapter~\ref{ch:future} describes what
can be improved in the future and Chapter~\ref{ch:conclusion} provides a
resume and conclusion to this dissertation.

% kate: default-dictionary en_GB; indent-width 2; replace-tabs on;
% kate: remove-trailing-space on; space-indent on;
% kate: replace-trailing-space-save on; remove-trailing-space on;
