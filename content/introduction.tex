\chapter{Introduction}

% Explain the problem

In human languages, it is possible that the same word has different meanings
according to the context in which it is used. For example:

\begin{enumerate}[label=(\alph*)]
 \item She shot the arrow using her \textit{bow}.
 \item He tied the \textit{bow} of the gift.
\end{enumerate}

In these two sentences, the word \textit{bow} means different things. In
sentence (a), it is used as a hunting tool, while in sentence (b), it is a 
ribbon. To be able to do some tasks in \ac{NLP}, machines need to be able to
differentiate between the different meaning of the same word in each use.

% Show why the problem is ubiquitous

This problem affects, among other \ac{NLP} tasks, \ac{MT}, \ac{IR} and content
categorization \citep{navigli2009word}. Taking an example from \ac{MT}, the
Portuguese word \textit{laço} can be translated in various different English
words depending on context. It can be translated into \textit{ribbon} or it can
be translated as the \textit{bond} between two people.

\begin{enumerate}[label=(\alph*)]
	\item Ela atou o presente com um \textbf{laço}. \\ \textit{She tied the 
	gift with a ribbon.}
	\item Há um forte \textbf{laço} entre eles. \\ \textit{There is a strong 
	bond between them.}
\end{enumerate}

A system which naïvely uses the \ac{MFS} of a given word, as found in a 
semantically annotated \emph{corpus}, may improperly translate or categorize 
this word when it is used outside of its most common context or in a different 
corpus.

% Introduce the solution to the problem

To deal with this problem, it is necessary to identify the specific meaning of
a word based on its context. This process is called \ac{WSD}
\citep{navigli2009word}. It requires both sense inventories and large amounts 
of sense-tagged \emph{corpora} to function efficiently. As a result, 
under-resourced languages need to deal with greater hardships to be
able to achieve satisfactory results \citep{ng1997getting}.

A solution to the lack of resources is to automatically identify the meaning of
words in their given context, without the requirement of manually annotated
data. This is called \ac{WSI} \citep{agirre2007semeval}. Most \ac{WSI} models 
rely on word co-occurrence to determine the main senses a word may have. 
Syntactic dependencies between words are seldom used.

% Propose questions which this paper will answer

% Is it possible?
% Does it execute within reasonable limits?
% Can it be better than no induction/disambiguation?
% How do you evaluate it? Aka, how do you know it is better?

The goal of this dissertation is to investigate the feasibility of creating a
\ac{WSI} model for the Portuguese language, which would be capable of using 
syntactic dependency information to determine the main senses a word may have. 
Furthermore, this dissertation looks into evaluating the quality 
of this new model against the \ac{MFS} baseline.

% Introduce the shiny new tool, explain the problem it solves

Additionally, this dissertation presents the results of this investigation in
a project called \ac{SENTIDO}. \ac{SENTIDO} is a \ac{WSI} and \ac{WSD} model
which infers the possible senses of a word from sense-untagged \emph{corpora} based its syntactic relations with its context; and, given a word and its context, the system disambiguates between the previously inferred senses.

% Describe the layout of this paper

This dissertation is organized as follows: In Chapter~\ref{ch:stateofart}, existing \ac{WSI} implementations and models are described, as well and the theoretical foundations and additional tools that support and aide them. In Chapter~\ref{ch:architecture}, the architecture of the model is outlined, as well as the various stages that compose it. In Chapter~\ref{ch:implementation}, the implementation details, such as the \emph{corpora} used, existing tools from which the model is developed, and adaptations made to those same tools, are depicted. Chapter~\ref{ch:eval} describes how the algorithm is tested, which are the theoretical foundations for the soundness of the methodology, the test corpus used, and the chosen parameters and finally, this chapter analyses the results of the evaluation. Chapter~\ref{ch:conclusion} provides a synopsis of the findings and describes what can be improved in the future.

% kate: default-dictionary en_GB; indent-width 2; replace-tabs on;
% kate: remove-trailing-space on; space-indent on;
% kate: replace-trailing-space-save on; remove-trailing-space on;
