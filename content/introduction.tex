\chapter{Introduction}

\section{Motivation}

In human languages, it is possible that the same word has different meanings
according to the context in which it is used. For example:

\begin{enumerate}[label=(\alph*)]
 \item She shot the arrow using her \textit{bow}.
 \item He tied the \textit{bow} of the gift.
\end{enumerate}

In these two sentences, the word \textit{bow} means different things. In
sentence (a) it is used as a hunting tool, while in sentence (b) it is a ribbon.
To be able to do some tasks in \ac{NLP}, machines need to be able to
differentiate between each use of the same word. \ac{WSD} is thus the
identification of the underlying meaning of words in their context
\cite{navigli2009word}.

\ac{WSD} is a task which is required both inside and outside the field of
\ac{NLP}. Machine translation, \ac{IR}, and all kinds of content categorization
benefit from \ac{WSD} to reduce uncertainty when preforming their goals. In
machine translation, for example, a word in Portuguese -- \textit{laço} -- might
be possible to translate to different words in English according to its context
-- \textit{ribbon} or \textit{bond} between two people \cite{navigli2009word}.
As publication of content increases exponentially \cite{james2016data}, new
uses for \ac{WSD} and places where using \ac{WSD} might benefit the quality of
the results are constantly being found and discovered.

\ac{WSD} systems require sense inventories and sense-tagged corpora to function.
Without these resources, preforming \ac{WSD} is not possible, and the systems
which could make use of word senses need to work with degraded precision,
normally by assuming the \ac{MFS} the word takes on the corpora \cite{TODO}.
To be able to preform \ac{WSD} in these under-resourced languages, it is
necessary to manually assemble the sense inventories and to review corpora so as
to tag senses into its words \cite{ng1997getting}.

\ac{WSI} is the task of automatically identifying the meanings of words
without presence of manually annotated data \cite{agirre2007semeval}. Many
systems have been implemented to perform \ac{WSI}, some of which
will be later described in Section~\ref{ch:stateofart}.

\section{Objectives}

\ac{SENTIDO}'s goal is to create an architecture capable of generating
labels for a word which are related to the word's semantic meanings, using the
context in a corpus to preform such task. Additionally, it intends to provide a
mechanism, given a word, its induced labels, and a context, to disambiguate
which label most closely corresponds to the usage of the word in this context.


% kate: default-dictionary en_GB; indent-width 2; replace-tabs on;
% kate: remove-trailing-space on; space-indent on;
% kate: replace-trailing-space-save on; remove-trailing-space on;
