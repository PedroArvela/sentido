\documentclass[a4paper,10pt,twoside,onecolumn,final,openright,leqno]{book}
\usepackage[doublespace,final]{thesis}
\usepackage{ist-thesis}
%\usepackage[portuguese,english]{babel}
\usepackage[utf8]{inputenc}
\usepackage[T1]{fontenc}
\usepackage[en-GB]{datetime2}

\usepackage{acronym}
\usepackage{epstopdf}
\usepackage{fancyvrb}
\usepackage{url}
\usepackage[hidelinks,bookmarks=true]{hyperref}

\title{SENTIDO: A Word Sense Induction Model for Portuguese}
\author{José Pedro de Almeida Arvela}
\field{Computer Science and Engineering}
\supervisor{Prof. Nuno João Neves Mamede,\\ Prof. Jorge Manuel Evangelista
Baptista}

\chairperson{Unknown}
\committeesup{Unknown}
\committeemembers{Unknown}

\DTMlangsetup{showdayofmonth=false}

\begin{document}

\maketitle

\begin{acknowledgements}
 TODO
\end{acknowledgements}

\begin{dedications}
 TODO
\end{dedications}

\begin{resumo}
 TODO
\end{resumo}

\begin{abstract}
 TODO
\end{abstract}

\begin{keywords}
 TODO

 TODO
\end{keywords}


\tableofcontents

\listoffigures
\newpage

\listoftables
\newpage

\begin{aclist}
  \acro{NLP}{Natural Language Processing}
\end{aclist}

% Start the real document
\startdocument

% Input files to use here: input{}

% Bibliography
\bibliographystyle{chicago}
\bibliography{refs}

\end{document}
% kate: default-dictionary en_GB; indent-width 2; replace-tabs on;
% kate: remove-trailing-space on; space-indent on;
% kate: replace-trailing-space-save on; remove-trailing-space on;
